\documentclass[graphics]{beamer}

\usepackage{graphicx}
\usepackage{verbatim}
\usepackage{wrapfig}
\useoutertheme{shadow}
%\usecolortheme{orchid}
\usecolortheme{seahorse}


% math commands
\newcommand{\be}{\begin{eqnarray}}
\newcommand{\ee}{\end{eqnarray}}
\newcommand{\beq}{\begin{equation}}
\newcommand{\eeq}{\end{equation}}
\def\simless{\mathbin{\lower 3pt\hbox
      {$\rlap{\raise 5pt\hbox{$\char'074$}}\mathchar"7218$}}}
\def\simgreat{\mathbin{\lower 3pt\hbox
      {$\rlap{\raise 5pt\hbox{$\char'076$}}\mathchar"7218$}}} %> or of order

% variables

\def\toonscale{0.45}
\def\mboxy#1{\mbox{\small #1}}


\begin{comment}
\AtBeginSection[]{
  \frame{
    \frametitle{Outline}
    \tableofcontents[currentsection]
  }
}
\end{comment}

\title{Vision for ASIAA
}
%\subtitle{interim update}
\author[U. Pen]{Ue-Li Pen, 
\\[8mm] 
}
\date{January 16, 2020}


\begin{document}

%\section*{Introduction}
\section{Background}

\begin{comment}
  \subsection{Outline}

  \frame{
    \frametitle{Outline}
    \tableofcontents
  }
\end{comment}

\frame{\maketitle}



  \frame{
    \frametitle{Background}
    \begin{itemize}
        \item Global background: Germany, Taiwan (NTU, NCTU), US
          (Princeton, Harvard), Canada (CITA), with
          extended research in China, India
        \item served as CITA director 2016-2019, on various boards
          including NRC-Herzberg, computing
    \end{itemize}
%\vspace{-0.2in}
  }




  \frame{
%\vspace{-0.5in}
    \frametitle{Project Highlights}
    \begin{itemize}
    \item Theoretical Cosmology: topological defects (cosmic strings),
      hydro/N-body (including Tian-Nu, worlds largest N-body sim in
      China with 3 trillion particles)
    \item CMB: CBI, early AMIBA
    \item 21cm: PaST/21CMA, GMRT, GBT-IM, CHIME
    \item VLBI: Pulsar/FRB Scintillometry, EHT
    \end{itemize}
  }

\section{Theory}
  \frame{
%\vspace{-0.5in}
    \frametitle{Theory}
    \begin{itemize}
    \item Broad interests/portfolio
      \item track record in cosmology, compact objects, MHD, N-body
        \item build new linkages: galaxy spin (Jounghun Lee ++)
        \item within ASIAA and beyond (NTU, Earth/space Science, LC Lou)
    \end{itemize}
  }

  \frame{
\vspace{-0.5in}
    \frametitle{Strategy}
    \begin{itemize}
        \item close working relation with staff and project leaders
        \item strong relations to cognate domestic and international institutions:
        \item Universities in Taiwan: opportunities for astro faculty
          growth, joint graduate student
        \item AS: physics, earth science
        \item International: build on existing links (ALMA, Subaru,
          LSST, EAO, EHT) and identify new potential partners
          (e.g. MPI, NRC)
    \end{itemize}
  }


\section{VIsion}
  \frame{
\vspace{-0.5in}
    \frametitle{Vision}
    \begin{itemize}
    \item strengthen within, connect globally
        \item solidify resources with complementary strength
        \item mm-radio, OIR, Theory
        \item build on existing talent
        \item develop new opportunities
        \item potential example: mm FRB-VLBI gravitational lensing
    \end{itemize}
  }


  \frame{
%\vspace{-0.5in}
    \frametitle{FRB lensing}
    \begin{itemize}
    \item many (all?) FRBs repeat
        \item gravitational lensing creates coherent {\it echos}
          delayed by weeks
        \item each echo microlensed by stars and potentially DM
          delayed by microseconds
        \item coherent nano-second or pico-second time delay measurement
        \item after 1 year, the cosmic expansion/acceleration becomes
          1000 $\sigma$ effect
        \item need to study effects of lens proper motion, etc
        \item mm-VLBI allows avoids plasma effects, increases
          angular/time resolution
        \item matches across broad ASIAA strengths
    \end{itemize}
  }


\end{document}

% ref https://watermark.silverchair.com/sty2376.pdf?token=AQECAHi208BE49Ooan9kkhW_Ercy7Dm3ZL_9Cf3qfKAc485ysgAAAjYwggIyBgkqhkiG9w0BBwagggIjMIICHwIBADCCAhgGCSqGSIb3DQEHATAeBglghkgBZQMEAS4wEQQMqatTTvMLMeegjcbTAgEQgIIB6fWIwe_-r_37FxhDoB466CgjHacBLCPd9FLrsRKFDluvwG7uQqZvWpERoHuI0txiA8UjcCV7Ur6Erx7GnGzvBOxxghVqVMiKQJBr5gGUXC4mHBXFLAiVJ9I4p1fiZJexlkVv1DoS87Ps2jNzd1IswV7m2eSGzCKvOJetgVzLt9fDRD_pKf4hiPjXUHg2n6HrrWQfxPQhIwVWGodV7qdqoaJf21p9mZzgkf0DC1QfBbDL9nHq5hw9v9gn87amGpypYXaFekjU5Jr2zxaOVz8Be9J9fwi5ibk8btjayMsdT2j606m8cI65dZni1yJJuIkHNUT8WoAy5_l0J8s1la6OyovXTPEo66LdflI75_oMbh8eoNXrHm0GLKYTCA6csdLSA2UxuopXHhUraANRaM1UZ8mjOd-h1BcPmUgKcXXZtmEod54szQwMa7fLPX0VoJDAMT_I55tsJsDEoMIq1BWgShnxK5bgKPcl65GoguDjzf10FWCMcH3qtysJ3i4HZ4Lx9TzyI8eUEWilIGr3txiKxEnfE-g0wsNaY7icIx5U4PnsxYaU8tQKsCHQyNLlS1RgYYdzxl_kfStVNsLicP0dmCcWwpbAoSdWmewr-fZIc2NMrM2Y92GjgBn_lXDwICiy7hDTgNpjS4iMug
\documentclass[graphics]{beamer}

\usepackage{graphicx}
\usepackage{verbatim}
\usepackage{wrapfig}
\useoutertheme{shadow}
%\usecolortheme{orchid}
\usecolortheme{seahorse}


% math commands
\newcommand{\be}{\begin{eqnarray}}
\newcommand{\ee}{\end{eqnarray}}
\newcommand{\beq}{\begin{equation}}
\newcommand{\eeq}{\end{equation}}
\def\simless{\mathbin{\lower 3pt\hbox
      {$\rlap{\raise 5pt\hbox{$\char'074$}}\mathchar"7218$}}}
\def\simgreat{\mathbin{\lower 3pt\hbox
      {$\rlap{\raise 5pt\hbox{$\char'076$}}\mathchar"7218$}}} %> or of order

% variables

\def\toonscale{0.45}
\def\mboxy#1{\mbox{\small #1}}


\begin{comment}
\AtBeginSection[]{
  \frame{
    \frametitle{Outline}
    \tableofcontents[currentsection]
  }
}
\end{comment}

\title{Neutrino Torques
}
\subtitle{}
\author[U. Pen]{\textcolor{green}{Ue-Li Pen, H. Yu, X. Wang}
\\[8mm] 
}
\date{October 26, 2018}


\begin{document}

\frame{
\begin{picture}(320,250)
\put(-50,-130){
\includegraphics[width=5.5in]{Figures/delta_nu_sim.pdf}}
\end{picture}
\vspace{-3in}
\titlepage
}

%\section*{Introduction}
\section{Cosmological Neutrino Clustering}

\begin{comment}
  \subsection{Outline}

  \frame{
    \frametitle{Outline}
    \tableofcontents
  }
\end{comment}

  \frame{
\vspace{-0.5in}
    \frametitle{Neutrinos}
    \begin{itemize}
    \item defy expectations: existence, mass, oscillation, large
      mixing angles -- fruitful discvery space
        \item minimum mass 0.06 eV: $\Omega_\nu \sim 10^{-3} \ll \Omega_b$
        \item most massive neutrino non-relativistic today, affects LSS
        \item normal or inverted hierarchy?
        \item cosmological probes: how to disentangle such a small
          effect
        \item statistics not main problem: SDSS $1/\sqrt{n} \sim 10^{-3}$
%          \vspace{-0.15in}
    \end{itemize}
  }
  \frame{
    \frametitle{Observables}
    \begin{itemize}
        \item CMB, gravitational lensing: 2-D projection, 
        \item galaxies: 3-D biased displacement field
        \item Monge-Amp\'ere equation/solution
        \item ideally measure 2 fields, infer 2 fields: CDM, neutrinos (HDM)
    \end{itemize}
\vspace{-0.1in}\hspace{.3in}
\includegraphics[width=2.2in]{Figures/th2photo.jpg}
}
  \frame{
    \frametitle{Movie}
    {\tt http://cita.utoronto.ca/\~\,haoran/thnu/movie.html}
\includegraphics[width=4.2in]{Figures/thnucdmlowres.jpg}
}
  \frame{
    \frametitle{Galaxy Spins}
    \begin{itemize}
        \item most galaxies are rotating disks of stars and gas
        \item readily identifyable spin axis
        \item dust lanes, trailing spiral arms, HI velocity (rotation) field
     \end{itemize}
}
\frame{
    \frametitle{Observable}
\includegraphics[width=4.1in]{Figures/M51s.jpg}  

(M51, from Wikipedia)
  }

  \frame{
    \frametitle{Angular momentum}
    \begin{itemize}
        \item 1st order effect from misalignment of moment of inertia
          and tidal tensor
        \item $\tau\equiv\int \rho \bf{r} \times \nabla \phi$
        \item $\tau_i=\epsilon_{ijk} \int \rho x^jx^l \partial_l\partial_k\phi \equiv\epsilon_{ijk} I_{il}T_{lk}$
        \item $\tau= * I \cdot T$
        \item first realized by S. White (1984), see also LP00
     \end{itemize}
}


\frame{
    \frametitle{3-D: E-mode}
%\vspace{-0.5in}
\hspace{-0.2in}\includegraphics[width=2.3in]{Figures/nonlinear.png}  
\vspace{0.15in}\includegraphics[width=2.21in]{Figures/reconstructed.png}  

(from Yu et al 2016, 1610.7112)
  }

\frame{
    \frametitle{Lagrangian coordinates}
\center{\includegraphics[width=4.3in]{Figures/delta_reco_raw.pdf}  }
  }

  \frame{
    \frametitle{Coordinate freedom}
\begin{eqnarray}
{\rm potential\ deformation\ \ \ \ \ }  x^i &=& \xi^\mu \delta^i_\mu + \frac{\partial \phi}{\partial
    \xi^\mu}\delta^{i\mu}\nonumber\\
{\rm dreibein\ \ \ \ \ \  } e^i_\mu &\equiv& \partial x^i / \partial \xi ^ \mu \nonumber\\
 {\rm volume\  element\ \ \ \ }\sqrt{g} &\equiv& \mathrm{det}\left| e^i_\mu\right|\nonumber\\
{\rm mass\ coordinate \ \ \ \ \ }    \rho \sqrt{g}&=&\mathrm{Const.}\nonumber\\
    \partial _\mu (\rho \sqrt{g} e^\mu _i \delta^{i\nu}
    \partial_\nu \dot{\phi})&=&\langle\rho\rangle-\rho \sqrt{g}
\label{eqn:dif}
\end{eqnarray}
Solve Monge-Amp\'ere eqn (\ref{eqn:dif}) using multigrid (Pen 1995):
unique bijective mass coordinate.  See also Schmidtfull, Wang, Seljak,
Zaldarriaga ++
}

  \frame{
    \frametitle{Multigrid solution}
\vspace{-0.7in}\center{\includegraphics[width=4.0in]{Figures/map0512-0128_i1500_xz222.pdf}}
Zhu et al 1610.09638
}
  \frame{
    \frametitle{Redshift space}
\vspace{-0.7in}\center{\includegraphics[width=4.0in]{Figures/map0512-0128_i0900_xz222_rsd3.pdf}}
Zhu et al 1610.09638
}

  \frame{
    \frametitle{Low noise limit}
\includegraphics[width=3.4in]{Figures/rk.pdf}
}
  \frame{
    \frametitle{Halos}
\vspace{-0.5in}\hspace{-0.9in}\includegraphics[width=5.0in]{Figures/halocc.pdf}
}

 \frame{
    \frametitle{Predicting Neutrino Torques}
    \begin{itemize}
      \item $\Delta x (q) = q^j \partial_i\partial_j \psi$
        \item $I_c \sim T_c$: both describe particle displacement
        \item $j_\nu = \epsilon T_c T_\nu$
        \item Neutrino tidal torque is predictable observable from
          displacement potential
     \end{itemize}
  }

 \frame{
    \frametitle{Size estimate}
    \begin{itemize}
    \item $|j_\nu/j_c| \sim 10^{-4} (f_\nu /0.003) (\sqrt{P(k_{\rm FS})/P(k_{\rm vir})}/0.03$
    \item agrees with simulation measurement
    \item need $n> 10^8$ galaxy spins
    \item accessible in next generation 21cm surveys
     \end{itemize}
  }


  \frame{
\vspace{-0.5in}
    \frametitle{Future 21cm Surveys}
    \begin{itemize}
        \item expand on HSHS (Peterson et al 2006), CHIME
        \item build on economy of scale, map $10^{11}$ galaxies
     \end{itemize}
  }
  
\frame{
\vspace{-0.5in}
    \frametitle{More cosmological applications}
    \begin{itemize}
        \item map initial (linear) tidal field
        \item BAO, standard ruler (Alcock-Paczynski effect)
        \item modified gravity, time evolving neutrino mass
     \end{itemize}
  }


\frame{
\vspace{-0.5in}
    \frametitle{Conclusions}
    \begin{itemize}
      \item galaxy spins: new probe of initial conditions
      \item predictable from observable displacement field using
        non-linear reconstruction
      \item computationally straightforward, mass coordinate
            similar to Lagrangian
          \item already observable, scalable to much larger surveys
          \item enables measurement of 2 cosmic scalar fields:
            potential beat cosmic variance limits, etc
     \end{itemize}
  }


\end{document}

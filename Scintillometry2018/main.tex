\documentclass[graphics]{beamer}
\usepackage{xcolor}
\usepackage{graphicx}
\usepackage{verbatim}
\usepackage{wrapfig}
\usepackage{tabularx}
\usepackage{multirow}
\usepackage{amssymb}
\usepackage{pifont}
\usepackage{tikz}
\def\Checkmark{\tikz\fill[scale=0.2](0,.35) -- (.25,0) -- (1,.7) -- (.25,.15) -- cycle;} 

\useoutertheme{shadow}
%\usecolortheme{orchid}
\usecolortheme{seahorse}
\newcommand{\cmark}{\text{\ding{51}}}
%\newcommand*{\GtrSim}{\smallrel\gtrsim}

% math commands
\newcommand{\be}{\begin{eqnarray}}
\newcommand{\ee}{\end{eqnarray}}
\newcommand{\beq}{\begin{equation}}
\newcommand{\eeq}{\end{equation}}
\def\simless{\mathbin{\lower 3pt\hbox
      {$\rlap{\raise 5pt\hbox{$\char'074$}}\mathchar"7218$}}}
\def\simgreat{\mathbin{\lower 3pt\hbox
      {$\rlap{\raise 5pt\hbox{$\char'076$}}\mathchar"7218$}}} %> or of order

% variables

\def\toonscale{0.45}
\def\mboxy#1{\mbox{\small #1}}

\defbeamertemplate*{title page}{customized}[1][]
{
  \usebeamerfont{title}\inserttitle\par
  \usebeamerfont{subtitle}\usebeamercolor[fg]{subtitle}\insertsubtitle\par
  \bigskip
  \usebeamerfont{author}\insertauthor\par
  \usebeamerfont{institute}\insertinstitute\par
  \usebeamerfont{date}\insertdate\par
  \usebeamercolor[fg]{titlegraphic}\inserttitlegraphic
}
\begin{comment}
\AtBeginSection[]{
  \frame{
    \frametitle{Outline}
    \tableofcontents[currentsection]
  }
}
\end{comment}


\title{\textcolor{red}{Intro to Scintillometry}}
%\subtitle{}
\author[U. Pen]{{
{ 
\textcolor{green}{\small Ue-Li Pen}
}, 
\textcolor{red}{\small Toronto Scintillometry Collaboration} 
}
\\[8mm] 
\textcolor{black}{\small photo: Andre Renard}
}
\date{\textcolor{blue}{October 22, 2018}}



\begin{document}


%\includegraphics[width=4.4in]{Figures/IMG-7749-ARO-crop.JPG}

\frame{
\vspace{-0.5in}
\begin{center}  
%\includegraphics[width=4.4in]{Figures/IMG-0438-by-Andre-cropped.jpg}
\end{center}
\begin{picture}(320,250)
\put(-50,60){
\includegraphics[width=5.5in]{Figures/traverse-aurora.jpg}}
\end{picture}
\vspace{-4in}
\\
image credit: Andre Recnik
\\
\vspace{1in}
\titlepage
}


%\section*{Introduction}
\section{Introduction}

\begin{comment}
  \subsection{Outline}

  \frame{
    \frametitle{Outline}
    \tableofcontents
  }
\end{comment}

  \frame{
    \frametitle{Overview}
    \begin{itemize}
      \item ESEs, inverted arclets
      \item magnetism
      \item Scintillometry
      \item FRBs
      \item magnetars
      \item pulsars
    \end{itemize}
    \vspace{-1in}\hspace{2in}\includegraphics[width=0.5\textwidth]{Figures/IMG-7749-ARO-crop.JPG}
  }
  \frame{
    \frametitle{ESEs etc}
\hspace{-0.2in}\includegraphics[width=1.1\textwidth]{Figures/fiedler-fig.png}

Fiedler et al 1987
  }
  \frame{
    \frametitle{ESEs etc}
    \begin{itemize}
      \item Fiedler 1987++: chromatic radio flux variations
      \item AU transverse scale: if isotropic, requires $n_e\sim 10^3$cm$^{-3}$
      \item ISM explosions, or exotically confined
      \item exploding dark matter, quark nuggets, superconducting
        cosmic strings
      \item hints from pulsar inverted parabolic arclets
      \item lenses elongated, perhaps also along line of sight!
    \end{itemize}
  }



  \frame{
    \frametitle{Scintillometry}
PSR B0834+06: 

$D_S=620$pc, 

$D_L=389/415$pc

{\tiny Brisken+2010, Liu+2016}
\begin{picture}(320,250)
\put(110,90){
\includegraphics[width=0.7\textwidth]{Figures/Fig7_without_lines_5.png} 
}
\end{picture}
%\vspace{-4in}

  }

\frame{
    \frametitle{Lensing}
    \begin{itemize}
      \item geometric optics: refraction -- Snell's law
        $\frac{\sin\theta_1}{\sin\theta_2}=\frac{n_2}{n_1}$
      \item e.g. twinkling stars, eyeglasses
      \item multiple geometric images, that may interfere
      \item chromatic, $\Delta \theta\propto \lambda^2$
%      \item chromatic, $\Delta \theta\propto \lambda$
    \end{itemize}
%\vspace{-1.01in}
\hspace{1in}
\includegraphics[width=0.23\textwidth]{Figures/prism.png}
\includegraphics[width=0.23\textwidth]{Figures/Snells_law2.png}

%\vspace{-0.5in}
\tiny (from wikipedia)
%\vspace{0.5in}
%\vspace{0.9in}
}



\section{Lensing}



  \frame{
    \frametitle{Grazing incidence}
\includegraphics[width=0.9\textwidth]{Figures/toronto.png}
  }


\frame{
    \frametitle{Interference}
    \begin{itemize}
      \item Goldreich Sridhar 2006: refractive images generically
        interfere, e.g. double slit
      \item leads to scintillation scaling $\Delta \nu\propto \nu^{-4}$
      \item projected density caustics: Snell's law diverges
      \item statistics of alignment: rare alignments dominate
        lensing/scattering
      \item use ISM as giant billion km telescope!
    \end{itemize}

\tiny (from wikipedia)

\vspace{-0.5in}\hspace{2.55in}\includegraphics[width=0.4\textwidth]{Figures/Doubleslit.png}
}
  \frame{
    \frametitle{Current Sheets}
    \begin{itemize}
      \item magnetic field directional change is exact solution to MHD equilibrium
      \item stability unknown, meta-stable (Sweet-Parker 1957+) or unstable (tearing
        mode, Petscheck 1964, +++)
      \item proposed as source of scattering (Goldreich-Sridhar 2006,
        ULP-Levin 2014, Simard-ULP 2018)
    \end{itemize}
\includegraphics[width=0.9\textwidth]{Figures/reconnection.png} \tiny (Pang+2011)
  }

  \frame{
    \frametitle{Revisit}
\includegraphics[width=0.8\textwidth]{Figures/liu-lens.png} \tiny
Brisken+2010, Liu+Pen 2016
  }

  }
  \frame{
    \frametitle{Applications}
    \begin{itemize}
      \item cosmic telescope: picoarcsecond astrometry of magnetospheres
      \item measured 1km motion of PSR B0834+06 emission, initial results for crab
      \item potential for precision distances to pulsars, increased
        PTA sensitivity, accurate GW localization.
    \end{itemize}
%\vspace{-0.5in}\hspace{3.5in}\includegraphics[width=0.15\textwidth]{Figures/convergent_geometry.jpeg}
  }
 

  \frame{
    \frametitle{FRB}
    \begin{itemize}
    \item Masui+ 2015, FRB110523: lensing of tail by milky way localizes scattering screen to host
      galaxy, rules out IGM scattering
    \item  repeating FRB: potential to discriminate AGN from nebula
    \end{itemize}
  }

  \frame{
    \frametitle{Pulsars}
    \begin{itemize}
    \item map magnetospheres: first fringes on crab with GMRT-MWA,
      DRAO-ARO, EVN, radioastron (see talk by Lin)
    \item preliminary map of magnetosphere: pulse-interpulse widely
      separated, individual GP spatially resolved by nebula during
      strong scattering periods
     \item distances and inclinations to binary pulars: masses, sizes
       (inertia), distances
    \end{itemize}


\vspace{-0.3in}\hspace{1.8in} 
\includegraphics[width=0.4\textwidth]{Figures/allgate.pdf}
\vspace{0.5in}
\tiny Pen+ 2014

.
  }




  \frame{
    \frametitle{Discussion}
    \begin{itemize}
      \item ISM lenses: probed by compact radio sources at high frequencies
      \item pulsars are low frequencies
      \item ISM contains localized, 1-D, strong lensing screens
      \item Use ISM to map source structure
      \item Pulsars/FRB: unique source of coherent radiation
      \item 
    \end{itemize}
  }


  \frame{
    \frametitle{Conclusion}
    \begin{itemize}
      \item        plasma
        \item topologically constricted magnetic domains
      \item 1-D instead of 3-D lenses: potential to improve pulsar
        timing, mapping.
      \item potential to use scintillometry for precision pulsar
        distances, orbits, masses, improved PTA
    \end{itemize}
  }

\end{document}

\documentclass[graphics]{beamer}
\usepackage{xcolor}
\usepackage{graphicx}
\usepackage{verbatim}
\usepackage{wrapfig}
\usepackage{tabularx}
\usepackage{multirow}
\usepackage{amssymb}
\usepackage{pifont}
\usepackage{tikz}
\def\Checkmark{\tikz\fill[scale=0.2](0,.35) -- (.25,0) -- (1,.7) -- (.25,.15) -- cycle;} 

\useoutertheme{shadow}
%\usecolortheme{orchid}
\usecolortheme{seahorse}
\newcommand{\cmark}{\text{\ding{51}}}
%\newcommand*{\GtrSim}{\smallrel\gtrsim}

% math commands
\newcommand{\be}{\begin{eqnarray}}
\newcommand{\ee}{\end{eqnarray}}
\newcommand{\beq}{\begin{equation}}
\newcommand{\eeq}{\end{equation}}
\def\simless{\mathbin{\lower 3pt\hbox
      {$\rlap{\raise 5pt\hbox{$\char'074$}}\mathchar"7218$}}}
\def\simgreat{\mathbin{\lower 3pt\hbox
      {$\rlap{\raise 5pt\hbox{$\char'076$}}\mathchar"7218$}}} %> or of order

% variables

\def\toonscale{0.45}
\def\mboxy#1{\mbox{\small #1}}

\defbeamertemplate*{title page}{customized}[1][]
{
  \usebeamerfont{title}\inserttitle\par
  \usebeamerfont{subtitle}\usebeamercolor[fg]{subtitle}\insertsubtitle\par
  \bigskip
  \usebeamerfont{author}\insertauthor\par
  \usebeamerfont{institute}\insertinstitute\par
  \usebeamerfont{date}\insertdate\par
  \usebeamercolor[fg]{titlegraphic}\inserttitlegraphic
}
\begin{comment}
\AtBeginSection[]{
  \frame{
    \frametitle{Outline}
    \tableofcontents[currentsection]
  }
}
\end{comment}


\title{\textcolor{white}{Cosmic Microscopes}}
%\subtitle{}
\author[U. Pen]{{
\textcolor{green}{\small Ue-Li Pen, CITA}
}
\\[8mm] 
}
\date{\textcolor{green}{October 6, 2020}}


\begin{document}

\frame{
\vspace{-0.5in}
\begin{center}  
%\includegraphics[width=4.4in]{Figures/IMG-0438-by-Andre-cropped.jpg}
\end{center}
\begin{picture}(320,250)
\put(-35,6){
\includegraphics[width=5.1in]{Figures/IMG-7749-ARO-crop.JPG}
}
\end{picture}
\vspace{-4in}
\\
%image credit: NRAO/AUI/NSF
\\
\vspace{1in}
\titlepage
}

%\section*{Introduction}
\section{Introduction}

\begin{comment}
  \subsection{Outline}

  \frame{
    \frametitle{Outline}
    \tableofcontents
  }
\end{comment}

  \frame{
    \frametitle{Overview}
    \begin{itemize}
      \item Coherent interference
      \item well behaved lenses
      \item pulsars and FRBs
    \end{itemize}
  }

  \frame{
    \frametitle{History}
    \begin{itemize}
      \item IPS, leads to pulsar discovery
      \item pulsars scintillate in ISM, interpreted as turbulence
      \item scintillation may resolve pulsar emission
        (Wolsczcan+Cordes 87, Gupta+99, Pen+14)
      \item challenge to place/map lens (Main+18)
    \end{itemize}
  }

\frame{
    \frametitle{Black Widow PSR B1957+20}
%     \hspace{-0.5in}
\vspace{-0.25in}\includegraphics[width=1.11\textwidth]{Figures/bwgeom.pdf}
}

 
\frame{
    \frametitle{Microscope}
     \vspace{-0.65in}\includegraphics[width=0.75\textwidth]{Figures/bwlens.pdf}
}

  \frame{
    \frametitle{Underappreciated: Coherence}
    \begin{itemize}
    \item most pulsars and FRB's in coherent/eikonal limit
    \item multi-path propagation due to gravitational and plasma
      lensing
    \item dominant (sole?) population of extragalactic coherent sources
      exhibiting interference (scintillation!)
    \item path length measured to $ \delta L\sim$ nanoseconds
    \item dimensionless strain $h=\delta L/L\sim 10^{-25}$: unique
      window on cosmic space-time metric (e.g. Yang+ 2017)
    \end{itemize}
  }

  \frame{
    \frametitle{Multi-path lensing}
    \begin{itemize}
    \item gravitational lensing:
    \item compact objects: stars, planets (unique window on
      extragalactic planets!), black holes, dark matter.  Time delay:
      milli-nano seconds, (Jow+ 2002.01570)
    \item dark matter, substructure, self-interaction: years-milli-seconds
    \item coherent interference measures time delay to nano-seconds
      (Wucknitz+ 2004.11643)
    \item will change from day to day due to cosmic expansion,
      acceleration, transverse velocity
    \end{itemize}
  }


  \frame{
    \frametitle{Multi-plane lensing}
    \begin{itemize}
    \item compact objects: stars, planets (unique window on
      extragalactic planets!), black holes, dark matter.  Time delay: milli-nano seconds
    \item dark matter, substructure, self-interaction: years-milli-seconds
    \item coherent interference measures time delay to nano-seconds
    \item guaranteed to change from day to day
    \end{itemize}
  }

  \frame{
    \frametitle{path integrals}
    \begin{itemize}
    \item Huygens' principle: sum over all paths (same as quantum field
      theory: Fermat, Feynmann, Fresnel-Kirchhoff)
    \item highly oscillatory integrals $\int \exp[i \phi({\bf x})]d^nx$
    \item considered barely tractible in 2-d
    \item Picard-Lefshitz theory
    \end{itemize}
  }



\frame{
    \frametitle{Lens notation}
     \includegraphics[width=1.01\textwidth]{Figures/lens-schematic.png}
{\hspace{-0.1in}\tiny Feldbrugge+ 2019}
}

\frame{
    \frametitle{Geometric Lens}
\begin{center}
     \includegraphics[width=0.45\textwidth]{Figures/GeometricOptics.pdf}
\end{center}
}
\frame{
    \frametitle{Wave Lens}
     \includegraphics[width=1.01\textwidth]{Figures/wave-cusp.png}
}

\frame{
    \frametitle{Data}
     \includegraphics[width=0.38\textwidth]{Figures/b1957-lens.png}
\vspace{-0.4in}
     \includegraphics[width=0.52\textwidth]{Figures/frb-repeat.png}
{\tiny \hspace{-0.4in} Main et al 2018, CHIME 2019}
}


\section{FRBs}


\frame{
    \frametitle{FRB110523}
     \includegraphics[width=0.8\textwidth]{Figures/nature15769-f1.jpg}

Masui et al 2015
}
\frame{
    \frametitle{Scattering}
     \includegraphics[width=0.8\textwidth]{Figures/nature15769-sf2.jpg}
}


\frame{
    \frametitle{Crab}
     \includegraphics[width=1.1\textwidth]{Figures/TwoScreenGeometry.png}

(figure credit: R. Main)

$\tau_{\rm nebula}\gg \tau_{\rm ISM}$
}

  \frame{
    \frametitle{Two screen physics}
    \begin{itemize}
      \item crab: 
      \item  long $\sim 1$ms: local to pulsar (SNR)
      \item short $\sim 1 \mu$s: galactic
      \item FRB110523:
      \item long $\sim 1$ms: local to host
      \item short $\sim 1 \mu$s: galactic
      \item FRB121102:
      \item long $\sim 10$ns: local to host
      \item short $\sim 20 \mu$s: galactic
    \end{itemize}
  }
  \frame{
    \frametitle{interpretation}
    \begin{itemize}
    \item ms scattering is generally due multipath propagation
    \item location has been proposed in IGM (theoretically challenging) or intervening halos
    \item FRB110523 shows $\mu$s scintillation from Galactic multipath
    \item scattering tail scintillates!
    \item {\it stars twinkle, planets don't}
    \item constrains source size less than $\sim$ microarcsecond
    \item scattering effectively increases source size
    \item scattering screen is physically associated with FRB, not
      intergalactic or intervening. Rules out Vedentham+Phinney (1811.10876)
    \end{itemize}
  }

  \frame{
    \frametitle{Future possibilities}
    \begin{itemize}
    \item event rate proportionate to field of view, collecting area
    \item millions of bright FRBs per year, likely billions per year
      accessible with achievable budgets
    \item outriggers will localize all events at milli arcsecond resolution
    \item resolve some gravitational/plasma lenses
    \end{itemize}
  }

\section{Summary}



  \frame{
    \frametitle{Conclusion}
    \begin{itemize}
    \item coherence of pulsars+FRBs underappreciated
    \item one of the most precise measurements ($10^{-25}$)
    \item wave propagation poses new theory challenges: oscillatory
      path integrals (Feldbrugge at al 2019+)
      \item Two screen scintillation/scattering: crab, FRB110523
      \item ISM scintillation unlikely due to turbulence
      \item low frequency VLBI could quantify galactic screen distance,
        constrain source screen distance
      \item beginning of new era, for theory and data
    \end{itemize}
  }

\end{document}

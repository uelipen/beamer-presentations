\documentclass{beamer}
\usepackage{media9}
%\usepackage{movie15}


%\usepackage{graphicx}
\usepackage{verbatim}

\useoutertheme{shadow}
%\usecolortheme{orchid}
\usecolortheme{seahorse}

\renewcommand{\a}{\alpha}
\renewcommand{\b}{\beta}
\renewcommand{\d}{\delta}
\newcommand{\g}{\gamma}
\newcommand{\s}{\sigma}
\newcommand{\w}{\omega}
\renewcommand{\k}{\vec{k}}
\newcommand{\td}[1]{\tilde{#1}}
\newcommand{\x}{\vec{x}}
\newcommand{\p}{\phantom{\alpha}}

\def\toonscale{0.45}
\def\mboxy#1{\mbox{\small #1}}


\begin{comment}
\AtBeginSection[]{
  \frame{
    \frametitle{Outline}
    \tableofcontents[currentsection]
  }
}
\end{comment}

\title{21cm absorption}
\subtitle{
Galileo, small galaxies, AGN, dark energy
}
\author[Boyle and Pen]{Ue-Li Pen, Haoran Yu ++ \\[8mm] 
}
\date{CIRADA 2018, June 21, 2018}


\begin{document}

\frame{\titlepage}

%\section*{Introduction}
\section{Introduction}

\begin{comment}
  \subsection{Outline}

  \frame{
    \frametitle{Outline}
    \tableofcontents
  }
\end{comment}

  \subsection{Galileo}

  \frame{
    \frametitle{RFI}
    \begin{itemize}
    \item for lines of width $\lesssim 70$km/s doppler shift annually,
      easier to disentagle from RFI, synchrotron foregrounds
    \item effect decreases at high ecliptic coordinates
    \end{itemize}
    }

  \frame{
    \frametitle{DLA}
    \begin{itemize}
      \item find 21cm damped lyman absorber systems: typically systems of
        cold absorbers.
      \item most 21cm detections from follow up of optically selected candidates
      \item very limited samples of radio selected DLA (2 in last 60 years), publication
        bias.
      \item constraints on fine structure evolution, ISM temperature evolution
    \end{itemize}
    }

  \frame{
    \frametitle{AGN}
    \begin{itemize}
    \item associated absorbers: circum nuclear flows
    \item substantial fraction of radio sources in CHIME redshift
      range
    \item many may be broad, challenging to detect earth doppler
      motion (Curran et al 2016, 1608.01055)
    \end{itemize}
    }

  \frame{
    \frametitle{Follow-up}
    \begin{itemize}
    \item Yu et al 2014 (PRL, 113, 041303)
    \item $10^4-10^5$ sources: major follow-up challenge
    \item in band: GMRT?
    \end{itemize}
    }

  \frame{
    \frametitle{Dark Energy}
    \begin{itemize}
      \item Darling 2012: accuracy dominated by 3C286
      \item direct measure the increase in absorber redshifts:
        Sandage-Loeb effect
      \item monitor sky for 5-10 years, look for systematic increase
        (decrease) of absorber redshift with time
      \item copernican principle (homogeneity) constraints currently
        very weak, direct test of FRW metric
      \item precision dominated by narrow systems
      \item drives specs for fine frequency resolution
    \end{itemize}
    }

  \subsection{Conclusions}
  \frame{
    \frametitle{Conclusions}
    \begin{itemize}
    \item will increase 21cm absorber inventory by orders of magnitude
    \item improve understanding of DLA, AGN
    \item cosmological constraints on homogeneity, FRW, dark energy
    \end{itemize}
  }
\end{document}

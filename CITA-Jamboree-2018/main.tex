\documentclass[graphics]{beamer}
\usepackage{xcolor}
\usepackage{graphicx}
\usepackage{verbatim}
\usepackage{wrapfig}
\usepackage{tabularx}
\usepackage{multirow}
\usepackage{amssymb}
\usepackage{pifont}
\usepackage{tikz}
\def\Checkmark{\tikz\fill[scale=0.2](0,.35) -- (.25,0) -- (1,.7) -- (.25,.15) -- cycle;} 

\useoutertheme{shadow}
%\usecolortheme{orchid}
\usecolortheme{seahorse}
\newcommand{\cmark}{\text{\ding{51}}}
%\newcommand*{\GtrSim}{\smallrel\gtrsim}

% math commands
\newcommand{\be}{\begin{eqnarray}}
\newcommand{\ee}{\end{eqnarray}}
\newcommand{\beq}{\begin{equation}}
\newcommand{\eeq}{\end{equation}}
\def\simless{\mathbin{\lower 3pt\hbox
      {$\rlap{\raise 5pt\hbox{$\char'074$}}\mathchar"7218$}}}
\def\simgreat{\mathbin{\lower 3pt\hbox
      {$\rlap{\raise 5pt\hbox{$\char'076$}}\mathchar"7218$}}} %> or of order

% variables

\def\toonscale{0.45}
\def\mboxy#1{\mbox{\small #1}}

\defbeamertemplate*{title page}{customized}[1][]
{
  \usebeamerfont{title}\inserttitle\par
  \usebeamerfont{subtitle}\usebeamercolor[fg]{subtitle}\insertsubtitle\par
  \bigskip
  \usebeamerfont{author}\insertauthor\par
  \usebeamerfont{institute}\insertinstitute\par
  \usebeamerfont{date}\insertdate\par
  \usebeamercolor[fg]{titlegraphic}\inserttitlegraphic
}
\begin{comment}
\AtBeginSection[]{
  \frame{
    \frametitle{Outline}
    \tableofcontents[currentsection]
  }
}
\end{comment}


\title{\textcolor{red}{Natural Cosmic Telescopes}}
%\subtitle{}
\author[U. Pen]{{
{ 
\textcolor{green}{\small R. Main, D. Simard, D. Baker, F. Lin,
  A. Roman, A. Patil, 
F. Kirsten, I. Yang, V. Marthi}
}, 
\textcolor{red}{\small M. van Kerkwijk, K. Vanderlinde, JP Macquart,
  U. Pen} 
and more scintillometers
}
\\[8mm] 
}
\date{\textcolor{blue}{September 27, 2018}}


\begin{document}


%\includegraphics[width=4.4in]{Figures/IMG-7749-ARO-crop.JPG}

\frame{
\vspace{-0.5in}
\begin{center}  
%\includegraphics[width=4.4in]{Figures/IMG-0438-by-Andre-cropped.jpg}
\end{center}
\begin{picture}(320,250)
\put(-50,60){
\includegraphics[width=5.5in]{Figures/traverse-aurora.jpg}}
\end{picture}
\vspace{-4in}
\\
image credit: Andre Recnik
\\
\vspace{1in}
\titlepage
}


%\section*{Introduction}
\section{Introduction}

\begin{comment}
  \subsection{Outline}

  \frame{
    \frametitle{Outline}
    \tableofcontents
  }
\end{comment}

  \frame{
    \frametitle{Overview}
    \begin{itemize}
      \item natural billion km sized plasma lenses enable nano
        arcsecond astrometry and mapping in radio
      \item lenses images by VLBI
      \item successful application to pulsars, FRBs: cross over of
        wave/geometric optics
      \item new window to test pulsar/FRB emission theory.
      \item achieved $\sim$ mapping at kpc distance: $\sim 50$ pico
        arcseconds
      \item tools: CHIME, ARO, DRAO, HSA, etc, + HPC, 
    \end{itemize}
    \vspace{-1in}\hspace{2in}\includegraphics[width=0.5\textwidth]{Figures/IMG-7749-ARO-crop.JPG}
  }


  \frame{
    \frametitle{Scattering}
    \begin{itemize}
      \item 'seeing', 'multi-path', 'blurring'
      \item twinkling of stars
      \item interplanetary scintilation, ionospheric disturbance
      \item usually viewed as degradation of signal
      \item once thought as stochastic, turbulent, volume filling,
        many degrees of freedom
      \item if understood, use as pico arcsecond telescope, microscope
      \item parallax distance, mass, precision gravitational wave localization
      \item FRB's provide most precise space-time probes.
    \end{itemize}
  }


  \frame{
    \frametitle{Scintillometry}
PSR B0834+06: 

$D_S=620$pc, 

$D_L=389/415$pc

{\tiny Brisken+2010, Liu+2016}

\includegraphics[width=0.6\textwidth]{Figures/liu-lens.png} \tiny
Brisken+2010, Liu+Pen 2016
  }

  \frame{
    \frametitle{Applications}
    \begin{itemize}
      \item cosmic telescope: picoarcsecond astrometry of magnetospheres
      \item measured 1km deflection of PSR B0834+06 emission, initial
        results for crab, black widow
      \item potential for precision distances to pulsars, increased
        PTA sensitivity, accurate GW localization.
    \end{itemize}
  }
 

\end{document}

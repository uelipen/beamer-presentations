\documentclass{beamer}
\usepackage{media9}
%\usepackage{movie15}


%\usepackage{graphicx}
\usepackage{verbatim}

\useoutertheme{shadow}
%\usecolortheme{orchid}
\usecolortheme{seahorse}

\renewcommand{\a}{\alpha}
\renewcommand{\b}{\beta}
\renewcommand{\d}{\delta}
\newcommand{\g}{\gamma}
\newcommand{\s}{\sigma}
\newcommand{\w}{\omega}
\renewcommand{\k}{\vec{k}}
\newcommand{\td}[1]{\tilde{#1}}
\newcommand{\x}{\vec{x}}
\newcommand{\p}{\phantom{\alpha}}

\def\toonscale{0.45}
\def\mboxy#1{\mbox{\small #1}}


\begin{comment}
\AtBeginSection[]{
  \frame{
    \frametitle{Outline}
    \tableofcontents[currentsection]
  }
}
\end{comment}

\title{PTA Confusion}
\subtitle{Resolution, Source Confusion and Stochasticity in Pulsar Timing Arrays}
\author[Boyle and Pen]{Ue-Li Pen, Latham Boyle ++ \\[8mm] 
}
\date{IPTA 2018, June 22, 2018}


\begin{document}

\frame{\titlepage}

%\section*{Introduction}
\section{Introduction}

\begin{comment}
  \subsection{Outline}

  \frame{
    \frametitle{Outline}
    \tableofcontents
  }
\end{comment}

  \subsection{PTA}
  \frame{
    \frametitle{Redshift maps}
    \begin{itemize}
	\item Boyle+UP 2012, Roebber+Holder 2017
	\item Densely sampled limit:
        \item ToA map: 3-D
        \item FFT into complex 2-D map at each frequency
    \end{itemize}
  }
  \frame{
    \frametitle{Redshift Image}
%    \begin{minipage}{0.6\linewidth}
    \vspace{-0.2in}
    \begin{itemize}
    \item \mboxy{$\delta t = \sin(2\phi)[1+\cos(\theta)]$}
    \end{itemize}
%    \end{minipage}
    \begin{minipage}{0.35\linewidth}
%      \includegraphics[scale=0.25]{Figures/residual}
%\visible{\includemovie[poster,text={\small(Loading
%    Video...)}]{6cm}{4cm}{Figures/residual.mp4}}

\includemedia[
  label=some_dice,
  width=3.0\linewidth,height=1.35\linewidth, % 16:9
  addresource=Figures/residual.mp4, 
  transparent,
  activate=pageopen,
  passcontext,
  flashvars={
    source=Figures/residual.mp4
   &autoPlay=true % start playing on activation
   &loop=true
  }
]{}{VPlayer.swf} 
% loop video
\mediabutton[
  mediacommand=some_dice:playPause,
  overface=\color{blue}{\fbox{\strut Play/Pause}},
  downface=\color{red}{\fbox{\strut Play/Pause}}
]{\fbox{\strut Play/Pause}}
\mediabutton[
  mediacommand=some_dice:setSource [(filename.mp4)]
]{\fbox{\strut 
}}


    \end{minipage}
    \hspace{0.02\linewidth}
    
        singular spatial structure near GW source, no residual in
    anti-direction (TT).  

    }

  \frame{
    \frametitle{Two sources}
      \vspace{0.5cm}
      \includegraphics[scale=0.35]{Figures/residual2}
    \begin{itemize}
      \item Well resolved by dense PTA
    \end{itemize}
    }



  \subsection{2-D}

  \frame{
    \frametitle{Background}
    \begin{itemize}
      \item traditional technique: measure 2 point correlation
        function of residuals (Hellings-Downs, Jenet, etc).
      \item equivalent to power spectrum
      \item potentially very non-optimal for point sources -- c.f.
        hubble deep field.
      \item new ``imaging'' approach
      \item initial detections will be most sensitive for
        $\lambda_{\rm GW} \sim t_{\rm obs}$.  We will study this limit.
      \item start with the limit of a dense sample of pulsars, and ask
        what happens as density is reduced
      \item should reduce to previously known limits w.r.t. confusion
        (e.g. Sesana et al 2008)
    \end{itemize}
  }

\section{Gravity Wave Image}

\subsection{Residual}
\frame{
  \frametitle{Formulation}
TT gauge line element:
\begin{equation}
  \label{metric}
  ds^{2}=-dt^{2}+[\delta_{ij}+2h_{ij}]dx^{i}dx^{j}.
\end{equation}
In this gauge, the $\vec{x}={\rm constant}$ worldlines are timelike
geodesics; along such a worldline, the proper time $\tau$ is just the
coordinate time $t$.   Single gravitational
plane wave travelling in the $\hat{n}$ direction
\begin{equation}
  \label{deltat(omega)}
  \delta \tilde{t}_{\alpha}(\omega)=\frac{i}{\omega}
  \frac{\tilde{h}_{ij}(\omega)\hat{r}_{\alpha}^{i}\hat{r}_{\alpha}^{j}
    [1-{\cal P}_{\alpha}(\omega)]}{(1\!+\!\hat{n}\!\cdot\!\hat{r}_{\alpha})}
\end{equation}
with phase
\begin{equation}
  \label{P_alpha}
  {\cal P}_{\alpha}(\omega)\equiv
  {\rm e}^{i\omega r_{\alpha}(1+\hat{n}\cdot\hat{r}_{\alpha})}.
\end{equation}
reduces to $\delta t = \sin(2\phi)[1+\cos(\theta)]$ when averaging
over all pulsar distances.  Well known result.
}

  \subsection{E-M Analogue}

  \frame{
    \frametitle{Radio Interferometers}
    \begin{itemize}
      \item Measures electric field $E_i$, v.s. strain $h_{ij}$
      \item Transverse wave
      \item Image (radiation power) detection is quadratic in $E$ (or $h$)
      \item stationary in time, fourier modes
      \item Source power spectrum is quadratic in image, quartic in field.
    \end{itemize}
  }


  \subsection{Finite Corrections}

  \frame{
    \frametitle{Distance}
    \begin{itemize}
    \item At angles $\theta < \sqrt{\frac{\lambda_{\rm GW}}{r}}$ the
      intrinsic pulsar delay cancels the earth delay
    \item Typical distances $r \sim$ kpc, $\lambda_{\rm GW} \sim 3 $ pc,
      $\theta \sim 5^o$
    \item Confused if more than 100's of sources, or more sources than
      pulsars. 
    \end{itemize}
  }
      

  \subsection{Statistics}

  \frame{
    \frametitle{Forecasts}
    \begin{itemize}
    \item Jenet et al  measure the spatial 2-PCF of timing
      residuals at fixed wavelength
    \item $\xi(\theta)=\langle \delta t(0) \delta t(\theta) \rangle$
      fixed at single source $\theta=0$
    \item expectation value of $\langle \xi \rangle = \frac{3}{2} x \log x-\frac{x}{4}+\frac{1}{2}$, where $x=[1-\cos(\theta)]/2$
    \item erases the singular structures
    \item then weight $\int \xi \langle \xi \rangle dx$
    \item only optimal if average strain is 20 times larger than peak
      strain.
    \item factor of 20 was assumed to be order unity
    \item Instead, fit for individual source gravity wave templates
    \end{itemize}    
  }

  \section{3-D imaging}
  \subsection{GW interferometry}
  \frame{
    \frametitle{3-D}
    \begin{itemize}
      \item if distances to pulsars are known to better than 1
        wavelength, source positions are known to $\delta \theta \sim
        \frac{\lambda}{D}$.
      \item complications when pulsar period changes over the 3-D extent of PTA
    \end{itemize}
  }

  \frame{
    \frametitle{Distance measurement}
    \begin{itemize}
    \item VLBI Scintillometry: demonstrated 10 pico arcsecond astrometry
      (Pen++2014)
    \item ongoing test in known systems
    \item most promising for binary systems
    \item low frequency VLBI monitoring: LOFAR, GMRT, LWA, MWA, etc
    \end{itemize}
  }

  \subsection{Conclusions}
  \frame{
    \frametitle{Conclusions}
    \begin{itemize}
    \item Confusion spatial resolution of 2-D PTA: $<$ 90 degrees for
      all cases, $< 5$ degrees for dense sampling. Timing residual map
      has rich spatial structure, which is erased in correlation
      function analysis.
    \item 3-D: $\sim \frac{10'}{\rm SNR}$
    \item Changes physical interpretation of PTA GW signals: unlikely
      to be in stochastic regime.  2-PCF confusion analysis is
      self-fulfilling prophecy.
    \item PTA has potential to be high resolution GW telescope
    \item motivation for precision pulsar VLBI paralax
    \item  cylinder transit surveys  may accelerate pulsar search
      (c.f. Kaspi, Ransom)
    \end{itemize}
  }
\end{document}

\documentclass[graphics]{beamer}

\usepackage{graphicx}
\usepackage{verbatim}
\usepackage{wrapfig}
\useoutertheme{shadow}
%\usecolortheme{orchid}
\usecolortheme{seahorse}
\usepackage{tikzsymbols}
\usepackage{textcomp}
\usepackage{parskip}

% math commands
\newcommand{\be}{\begin{eqnarray}}
\newcommand{\ee}{\end{eqnarray}}
\newcommand{\beq}{\begin{equation}}
\newcommand{\eeq}{\end{equation}}
\def\simless{\mathbin{\lower 3pt\hbox
      {$\rlap{\raise 5pt\hbox{$\char'074$}}\mathchar"7218$}}}
\def\simgreat{\mathbin{\lower 3pt\hbox
      {$\rlap{\raise 5pt\hbox{$\char'076$}}\mathchar"7218$}}} %> or of order

% variables

\def\toonscale{0.45}
\def\mboxy#1{\mbox{\small #1}}


\begin{comment}
\AtBeginSection[]{
  \frame{
    \frametitle{Outline}
    \tableofcontents[currentsection]
  }
}
\end{comment}

\title{Helical (Parity Odd) Initial conditions
}
\subtitle{}
\author[U. Pen]{\textcolor{green}{Ue-Li Pen, ASIAA, CITA}
\\[8mm] 
}
\date{September 5, 2022}


\begin{document}

\frame{
\begin{picture}(320,250)
\put(-50,0){
\includegraphics[width=5.5in]{Figures/helicalpasta.png}}
Schaller+12, Nature, 481, 261
\end{picture}
\vspace{-3in}
\titlepage
}

%\section*{Introduction}
\section{Helicity: probe of the early universe}

\begin{comment}
  \subsection{Outline}

  \frame{
    \frametitle{Outline}
    \tableofcontents
  }
\end{comment}

   \frame{
    \frametitle{Helicity}
    \begin{itemize}
        \item Macroscopic physics (gravity, E-M, etc) are parity symmetric
        \item any observed parity asymmetry direct window to intial conditions
        \item recent claims of detection (Hou+22, Philcox 22), upper
          bounds (Motloch+22)
        \item large, abstract space of parity odd configurations in
          4PCF, difficult to quantify covariances and look-elsewhere effect
        \item quadric estimator classification
        \item includes galaxy spin
        \item {\bf generation mechanisms }
     \end{itemize}
}

  \frame{
    \frametitle{Quadratic forms}
    \begin{itemize}
        \item analogy: CMB lensing
        \item deflection estimator: $d_i=T \partial_i \bar{T}$
          (Okamoto-Hu++03)
        \item shear estimator $\Gamma_{ij}=\partial_i T\partial_j T$
          (Zhu-ULP 2011.08251)
        \item 2 scalar ($\kappa,\gamma^E$), 1 pseudo scalar
          ($\gamma^B$) degree of freedom
          \item under lensing $\gamma^E=\kappa$ and $\gamma^B=0$
            \item $\gamma^B$ souced by multi-plane lensing
            \item $H=\gamma^E-\kappa$ sourced by tensor modes (Dodelson+02)
        \item lensing power is a 4PCF,  related to $g_{NL}$
        \item squeezed limit $L\ll l$ reduces 4PCF to E and B lensing
          power of $L$
        \item B power is parity-odd, not helical: excess power in B an
          indication of parity-odd  non-Gaussianity, distinct from
          helicity
\item 
     \end{itemize}
}

\frame{
    \frametitle{CMB}
\vspace{-0.15in}
\hspace{-0in}\includegraphics[width=4.2in]{Figures/plancklens.png}  

Planck 2018
  }

\frame{
    \frametitle{CMB-B}
\vspace{-0.15in}
\hspace{-0in}\includegraphics[width=2.8in]{Figures/plancklensb.png}  
Planck 2018
  }

  \frame{
%\vspace{-0.5in}
    \frametitle{Observables}
    \begin{itemize}
    \item 4PCF -- not intuitive due to large dimension, difficult to
      compute covariance (8PCF!)
    \item distinguish excess power in parity odd fields from helical
      power assymmetry!  Excess power is generic.
    \item galaxy spin
    \item magnetic field helicity
    \item Tidal:
    \item in analogy to B mode lensing map, create 4 helical
      pseudo-scalar maps
    \item vector, tensor helical auto and cross power for same helicity
    \item L-R cross correlation violated statistical isotropy
    \item 3 functions of $k$, huge reduction from 3-D function
    \end{itemize}
    }
       
  \frame{
%\vspace{-0.5in}
    \frametitle{Helicity Generation: Vectors}
    \begin{itemize}
    \item prescription: 
    \item 1. generate scalar Gaussian Random field
    \item 2. generate (L) helical vector displacement field
    \item 3. remap (1) using (2) as done for lensing
    \item this generations a stationary helical scalar field with
      parity odd 4PCF of large amplitude.
    \end{itemize}
    }

  \frame{
%\vspace{-0.5in}
    \frametitle{Helicity Generation: Tensors}
    \begin{itemize}
    \item cannot be done by displacement field: not enough d.o.f.
    \item 1. generate (L) helical TT tensor field $h_{ij}$
    \item 2. use $h_{ij}$ as metric to calculate geodesic distance
      between pairs of points $d(x_i,x_j)$
    \item 3. diagonalize covariance matrix $C_{ij}=\xi^{-1}d(x_i,x_j)$
    \item pick gaussian random numbers with variance of the
      eigenvalues
    \end{itemize}
    }



 
\frame{
    \frametitle{Measurement}
%\vspace{-0.5in}
\hspace{-0in}\includegraphics[width=2.9in]{Figures/spincorr.pdf}  
Motloch+ 2021
  }


  
\frame{
\vspace{-0.5in}
    \frametitle{Galaxy Spin Helicity}
    \begin{itemize}
    \item statistical isotropy and homogeneity allows for helicity asymmetry (e.g. weak force)
        \item NOT Goedel/Longo effect (``net $k$ independent left/right spin'')
        \item angular momentum measures twist of tidal tensor
        \item helicity measures twist projected along $k$ vector 
     \end{itemize}
  }


  
\frame{
\vspace{-0.5in}
    \frametitle{Helicity results}
    \begin{itemize}
    \item   $\mu_L &=& \(0.41 \pm 0.53\) \times 10^{-2}$  maximal left is allowed
      \item $\mu_R &=& \(1.99 \pm 0.53\) \times 10^{-2} $  maximal right is disfavoured!
      \item $\mu_- = \(-1.58 \pm 0.75\) \times 10^{-2}$  Parity symmetry is allowed \dSmiley % \DejaSans{ ☺}
     \end{itemize}
  }

  
\frame{
\vspace{-0.5in}
    \frametitle{Conclusions}
    \begin{itemize}
      \item helicity is a preserved non-Gaussianity not created
        by late time non-linearity
      \item tidal classification of helical non-Gaussianity: vector,
        tensor
      \item 
     \end{itemize}
  }


\end{document}

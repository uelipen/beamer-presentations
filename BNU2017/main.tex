\documentclass[graphics]{beamer}

\usepackage{graphicx}
\usepackage{verbatim}
\usepackage{wrapfig}
\useoutertheme{shadow}
%\usecolortheme{orchid}
\usecolortheme{seahorse}


% math commands
\newcommand{\be}{\begin{eqnarray}}
\newcommand{\ee}{\end{eqnarray}}
\newcommand{\beq}{\begin{equation}}
\newcommand{\eeq}{\end{equation}}
\def\simless{\mathbin{\lower 3pt\hbox
      {$\rlap{\raise 5pt\hbox{$\char'074$}}\mathchar"7218$}}}
\def\simgreat{\mathbin{\lower 3pt\hbox
      {$\rlap{\raise 5pt\hbox{$\char'076$}}\mathchar"7218$}}} %> or of order

% variables

\def\toonscale{0.45}
\def\mboxy#1{\mbox{\small #1}}


\begin{comment}
\AtBeginSection[]{
  \frame{
    \frametitle{Outline}
    \tableofcontents[currentsection]
  }
}
\end{comment}

\title{Why Matter?
}
\subtitle{}
\author[U. Pen]{\textcolor{green}{Turok, Loeb
}
\\[8mm] 
}
\date{November 10, 2017}


\begin{document}

\frame{
\begin{picture}(320,250)
\put(-50,-130){
\includegraphics[width=5.5in]{Figures/delta_nu_sim.pdf}}
\end{picture}
\vspace{-3in}
\titlepage
}

%\section*{Introduction}
\section{Cosmological Neutrinos}

\begin{comment}
  \subsection{Outline}

  \frame{
    \frametitle{Outline}
    \tableofcontents
  }
\end{comment}


\frame{
%vspace{-0.5in}
    \frametitle{Primordial Black Holes}
    \begin{itemize}
      \item would form from 5-$\sigma$ peaks (Bird++ 2016) at $T\sim$ 100
        MeV (Turok+ULP 2016)
      \item significant neutrino damping, possible pulsar timing signature
     \end{itemize}
 \includegraphics[width=3.0in]{Figures/whitedata_strain_SNR_qscan_v11Vhigh.jpg}
 }
\frame{
    \frametitle{image}
%\hspace{-1.1in}
\includegraphics[width=4.5in]{Figures/LIGO_blackhole.jpg}
 }

\frame{
\vspace{-0.5in}
    \frametitle{Cosmic shocks}
    \begin{itemize}
      \item generate $\mu$Hz gravitational waves
      \item potentially detectable in PTA's
     \end{itemize}
  }


\Frame{
\vspace{-0.5in}
    \frametitle{Matter asymmetry}
    \begin{itemize}
      \item why is more matter than anti-matter in universe?
      \item standard electroweak model violated baryon number through
        anomaly, B+L violated, B-L conserved
      \item Sacharov criteria: 1. B violation, 2. C, CP violation,
        3. non-thermal
      \item either before 200 GeV, perhaps L violation in neutrino sector,
        then converted to B through electro-weak anomaly
      \item during or after 200 GeV: phase transition
     \end{itemize}
  }

\frame{
\vspace{-0.5in}
    \frametitle{Electro-weak domain walls}
    \begin{itemize}
      \item generic in MSSM
      \item global symmetry in Higgs field, domain walls form at 200 GeV
      \item QCD instantons break symmetry, walls annihilate at 100 MeV
      \item 2000x scale factor for catalytic baryogenesis
      \item weakly inhomegeneous BBN
     \end{itemize}
  }

\frame{
\vspace{-0.5in}
    \frametitle{Catalytic explosions}
    \begin{itemize}
      \item Electroweak nucleation runaway for densities $>$ (100
        MeV)$^4$ (QCD scale)
      \item satisfied by neutron stars
      \item instabilities (Chandrasekar?) may lead to explosion
      \item Higgs vacuum decays to neutrinos
      \item signature in ICECUBE
     \end{itemize}
  }

\frame{
%\vspace{-0.5in}
    \frametitle{GW170817}
    \begin{itemize}
      \item GRB 170817A: neutron star merger, @40 Mpc
      \item unusual GRB signature, probably kilonova breakout, low
        lorentz factor
      \item no ICECUBE neutrinos detected
     \end{itemize}
\includegraphics[width=2.9in]{Figures/GW_Versus_Matter_STILL__CREDIT__Karan_Jani_Georgia_Tech.jpg}
  }

\frame{
\vspace{-0.5in}
    \frametitle{Conclusions}
    \begin{itemize}
      \item LIGO opens new window on universe
      \item maybe detected dark matter?

      \item baryon asymmetry?
     \end{itemize}
  }


\end{document}

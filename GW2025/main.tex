\documentclass[graphics]{beamer}

\usepackage{graphicx}
\usepackage{verbatim}
\usepackage{wrapfig}
\useoutertheme{shadow}
%\usecolortheme{orchid}
\usecolortheme{seahorse}


% math commands
\newcommand{\be}{\begin{eqnarray}}
\newcommand{\ee}{\end{eqnarray}}
\newcommand{\beq}{\begin{equation}}
\newcommand{\eeq}{\end{equation}}
\def\simless{\mathbin{\lower 3pt\hbox
      {$\rlap{\raise 5pt\hbox{$\char'074$}}\mathchar"7218$}}}
\def\simgreat{\mathbin{\lower 3pt\hbox
      {$\rlap{\raise 5pt\hbox{$\char'076$}}\mathchar"7218$}}} %> or of order

% variables

\def\toonscale{0.45}
\def\mboxy#1{\mbox{\small #1}}


\begin{comment}
\AtBeginSection[]{
  \frame{
    \frametitle{Outline}
    \tableofcontents[currentsection]
  }
}
\end{comment}

\title{PTA Imaging: resolution+polarization
}
%\subtitle{interim update}
\author[U. Pen]{Ue-Li Pen, Anna Tsai, Dylan Jow
}
\date{April 22, 2025}


\begin{document}

%\section*{Introduction}
\section{Lenses}

\begin{comment}
  \subsection{Outline}

  \frame{
    \frametitle{Outline}
    \tableofcontents
  }
\end{comment}

\frame{\maketitle}



  \frame{
    \frametitle{(Incoherent) Pulsar Timing Arrays (PTAs)}
    \begin{itemize}
      \item initial GW evidence in 2023 (nanoGrav++)
        \item currently from a few dozen nearby PSRs
        \item often described as ``stochastic'' background
        \item currently analyzed in ``incoherent'' mode: pulsar
          distance not used (and often not known)
        \item Number of pixels in incoherent array: 45!
   \end{itemize}
  }

  \frame{
    \frametitle{Redshift Maps}
    \includegraphics[width=4.01in]{Figures/roebber.jpg}
    
{\tiny complex image, c.f. Boyle+Pen 2012}
  }

   \frame{
\vspace{-0.25in}
    \frametitle{Sky Maps}
    \begin{figure}
      \begin{columns}
        \column{.8\linewidth}
\includegraphics[width=\textwidth]{Figures/IQUV.pdf}
        \column{.2\linewidth}
        \caption{
The adjoint $A^\dagger$ projection PSF of a purely right circularly polarized gravitational wave
point-source located at $\hat{n} = \hat{y}$.  Stokes IQUV real. 
 }
      \end{columns}
    \end{figure}
  }


  \frame{
    \frametitle{finite pulsars}
 \begin{figure}
      \begin{columns}
        \column{.7\linewidth}
    \includegraphics[width=\textwidth]{Figures/resolution_elements.pdf}
      \column{.3\linewidth}
    \caption{The number of resolution elements of a PTA as a function of number of pulsars.}
      \end{columns}
    \end{figure}
  
}


  \frame{
    \frametitle{Coherent Pulsar Timing Arrays (PTAs)}
    \begin{itemize}
    \item some pulsars (e.g. PSR J0437) have precise distance to
      within a wavelength
    \item resolution increases to $\lambda/D$: arc minutes
        \item full map of galaxy with SKA, orders of magnitude
          improvement in sensitivity, resolution
        \item Precision distances with Scintillometry (Baker++22)
        \item Coherent imaging GW telescope with arcsecond resolution
        \item no ``stochastic'' regime: small number of SMBH near merger
        \item counterparts of supermassive BH binaries, precise redshifts
    \end{itemize}
  }

   \frame{
\vspace{-0.25in}
    \frametitle{Coherent Sky Maps}
    \begin{figure}
      \begin{columns}
        \column{.8\linewidth}
\includegraphics[width=\textwidth]{Figures/IQUVwpulsardist.jpg}
        \column{.2\linewidth}
        \caption{
Sky map including pulsar term, $r\sim 3 \lambda$.  Real pulsar distances much larger. 
 }
      \end{columns}
    \end{figure}
  }




  \frame{
    \frametitle{Diffractive (evanescent) imaging}
    \begin{itemize}
        \item most galaxies too weak to form real gravitational lens images
        \item always form evanescent images through wave optics
        \item analogous to quantum mechanical tunelling
        \item diffractive angle $\theta \sim \lambda/D$
        \item $\lambda \sim$ pc, $D\sim 500$pc $\longrightarrow \theta
          \sim 0.1^o$
        \item dominated by edge-on spirals!
        \item flux of evanescent (complex, diffractive) images not supressed 
    \end{itemize}
  }

  \frame{
    \frametitle{Time delay $H_0$}
    \begin{itemize}
        \item stack diffractive GW flux from all edge-on galaxies
        \item distance from angle-delay relation $\tau \sim \theta^2      L$
        \item Shapiro delay neglible off-axis
        \item sensitivity in SKA/DSA era
    \end{itemize}
  }



  \frame{
    \frametitle{Diffractive/Evanescent Gravitational Wave Lensing}
\includegraphics[width=4.5in]{Figures/ptadiffraction.jpg}

Jow+ULP, PRL 134, April 4, 2025
  }


  \frame{
\vspace{-0.25in}
    \frametitle{Potential}

\includegraphics[width=2.5in]{Figures/ptasn.jpg}

signal-to-
noise in PTA experiment vs
a given precision on H0. The dashed line shows the effective
maximum precision that can be achieved due to the effect of
lens’ peculiar velocity. (Jow+ULP 25)

  }




  \frame{
\vspace{-0.5in}
    \frametitle{Lensing by Edge-on galaxies}
    \begin{itemize}
    \item most edge-on galaxies are strong lenses due to stellar
      column density
    \item only a few such lenses cataloged
    \item strong selection effects: extinction, small splitting angle
    \item potentially detectable in radio: QSOs, FRBs
    \end{itemize}
  }


  \frame{
\vspace{-0.25in}
    \frametitle{Edge-on Galaxy Lenses}

\includegraphics[width=4.5in]{Figures/unionslens.jpg}

Barrosco+24
  }


  \frame{
\vspace{-0.25in}
    \frametitle{Searching for Edge-on Galaxies}

\includegraphics[width=4.5in]{Figures/youichilens.png}

Umetsu, Ohyama, ULP+
  }




  \frame{
\vspace{-0.5in}
    \frametitle{Discussion}
    \begin{itemize}
    \item Eikonal effects applicable to compact radio sources,
      e.g. FRBs, pulsars
    \item full wave
effect dominates for long wavelengths as Fresnel scale is bigger then Einstein radius
    \item gravitational waves:  LIGO, LISA, PTA
    \item with SKA PTA parameters and favourable source geometries, $H_0$
      measurement at 0.1\% possible.
      \item Shapiro delay uncertainty neglible
    \end{itemize}
  }



  \frame{
%\vspace{-0.5in}
    \frametitle{Conclusions}
    \begin{itemize}
    \item PTA signals likely fully polarized, point image/source search in
      Stokes space.
    \item native PTA has 45 pixels: likely to identify individual sources
     \item wave optics changes nature of astrophysical observables: Coherent FRB/pulsar/GW radiation one of the potentially most
      precise measurements in physics
      \item PTA weak diffractive lensing may give new tool for Hubble
        Constant tension
      \item evanescent lensing/instantons
     \item at long wavelength, evanescent lensed images are
        unsuppressed
     \item large population of edge-on lenses: Euclid!
    \end{itemize}
  }

\end{document}

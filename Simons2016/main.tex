\documentclass[graphics]{beamer}

\usepackage{graphicx}
\usepackage{verbatim}
\usepackage{wrapfig}
\useoutertheme{shadow}
%\usecolortheme{orchid}
\usecolortheme{seahorse}


% math commands
\newcommand{\be}{\begin{eqnarray}}
\newcommand{\ee}{\end{eqnarray}}
\newcommand{\beq}{\begin{equation}}
\newcommand{\eeq}{\end{equation}}
\def\simless{\mathbin{\lower 3pt\hbox
      {$\rlap{\raise 5pt\hbox{$\char'074$}}\mathchar"7218$}}}
\def\simgreat{\mathbin{\lower 3pt\hbox
      {$\rlap{\raise 5pt\hbox{$\char'076$}}\mathchar"7218$}}} %> or of order

% variables

\def\toonscale{0.45}
\def\mboxy#1{\mbox{\small #1}}


\begin{comment}
\AtBeginSection[]{
  \frame{
    \frametitle{Outline}
    \tableofcontents[currentsection]
  }
}
\end{comment}

\title{Non-linear Reconstruction for 21cm IM
}
\subtitle{}
\author[U. Pen]{\textcolor{green}{Ue-Li Pen,
 H. Yu, H. Zhu, Q. Pan, X. Wang, Y. Yu, C. Kwan
}
\\[8mm] 
}
\date{December 20, 2016}


\begin{document}

\frame{
\begin{picture}(320,250)
\put(-30,-110){
\includegraphics[width=5.5in]{Figures/map0512-0128_i1500_.png}}
\end{picture}
\vspace{-3in}
\titlepage
}

%\section*{Introduction}


\begin{comment}
  \subsection{Outline}

  \frame{
    \frametitle{Outline}
    \tableofcontents
  }
\end{comment}

\section{21cm Intensity Mapping}
  \frame{
\vspace{-0.5in}
    \frametitle{Experiments}
    \begin{itemize}
      \item in memory: Fred Kwok-Yung Lo
        \item ongoing: GBT, Parkes, CHIME, Tianlai
        \item construction: Hirax, BINGO
        \item high-z: GMRT, LOFAR, MWA, Paper, (HERA, SKA)
     \end{itemize}
  }
\frame{
    \frametitle{CHIME}
%\vspace{-0.5in}
\hspace{-0.2in}\includegraphics[width=2.3in]{Figures/CHIME-2_1200px.jpg}  
\vspace{0.15in}\includegraphics[width=2.21in]{Figures/Ringmaps1.png}  
  }

  \frame{
\vspace{-0.5in}
    \frametitle{Reconstruction}
    \begin{itemize}
      \item 21cm IM maps potentially have very high density of objects
      \item optical trend has been to sample at low density to bypass
        non-linear effects
      \item success from Eisenstein++: displacement reconstruction
      \item signal loss, noise increase (Ngan et al 2012)
      \item Mass coordinates: Zhu et al 1609.07041
      \item reverse N-body: Tully-Peebles, Goldberg, Wang (ELUCID)
     \end{itemize}
  }

\frame{
\vspace{-0.5in}
    \frametitle{1-D}
    \begin{itemize}
        \item McQuinn\&White 2016
        \item unique mass coordinate
        \item closest non-parametric reconstruction (Zhu et al 2016)
     \end{itemize}
\includegraphics[width=4.21in]{Figures/gaussian_pert.pdf}  
  }
\frame{
    \frametitle{3-D: E-mode}
%\vspace{-0.5in}
\hspace{-0.2in}\includegraphics[width=2.3in]{Figures/nonlinear.png}  
\vspace{0.15in}\includegraphics[width=2.21in]{Figures/reconstructed.png}  

(from Yu et al 2016, 1610.7112)
  }

\frame{
    \frametitle{Lagrangian coordinates}
\center{\includegraphics[width=4.3in]{Figures/delta_reco_raw.pdf}  }
  }

  \frame{
    \frametitle{Low noise limit}
\begin{eqnarray}
{\rm potential\ deformation\ \ \ \ \ }  x^i &=& \xi^\mu \delta^i_\mu + \frac{\partial \phi}{\partial
    \xi^\mu}\delta^{i\mu}\nonumber\\
{\rm dreibein\ \ \ \ \ \  } e^i_\mu &\equiv& \partial x^i / \partial \xi ^ \mu \nonumber\\
 {\rm volume\  element\ \ \ \ }\sqrt{g} &\equiv& \mathrm{det}\left| e^i_\mu\right|\nonumber\\
{\rm mass\ coordinate \ \ \ \ \ }    \rho \sqrt{g}&=&\mathrm{Const.}\nonumber\\
    \partial _\mu (\rho \sqrt{g} e^\mu _i \delta^{i\nu}
    \partial_\nu \dot{\phi})&=&\langle\rho\rangle-\rho \sqrt{g}
\label{eqn:dif}
\end{eqnarray}
Solve diffusion eqn (\ref{eqn:dif}) using multigrid (Pen 1995)
}

  \frame{
    \frametitle{Multigrid solution}
\center{\includegraphics[width=3.0in]{Figures/map0512-0128_i1500_.png}}
Zhu et al 1610.09638
}

  \frame{
    \frametitle{Low noise limit}
\includegraphics[width=3.4in]{Figures/rk.pdf}
}
  \frame{
    \frametitle{BAO}
\includegraphics[width=3.4in]{Figures/dlnpk_dlns.png}

}
  \frame{
    \frametitle{FoM}
\includegraphics[width=3.4in]{Figures/BAOerr.png}

}
  \frame{
    \frametitle{Halos}
\vspace{-4in}\hspace{-0.5in}\includegraphics[width=5.0in]{Figures/map_both_150.pdf}

Voronoi tiling, Yu et al (2016, in prep)
}
  \frame{
    \frametitle{Halos}
\vspace{-0.5in}\hspace{-0.9in}\includegraphics[width=5.0in]{Figures/halocc.pdf}
}

  \frame{
\vspace{-0.5in}
    \frametitle{low z, low noise, 21cm Surveys}
    \begin{itemize}
        \item tunable: Parkes, Tianlai, BINGO
        \item Highjacking: Apertif, GMRT, ASKAP, SKA
     \end{itemize}
  }

  \frame{
\vspace{-0.5in}
    \frametitle{Science applications}
    \begin{itemize}
        \item BAO: maximize leverage from low-z where dark energy dominates
        \item neutrino wake dipole
        \item RSD
     \end{itemize}
  }


\frame{
\vspace{-0.5in}
    \frametitle{Conclusions}
    \begin{itemize}
        \item most precise non-parametric reconstruction mechanism to
          date
          \item computationally straightforward, mass coordinate
            similar to Lagrangian
        \item well suited for low shot noise low z BAO surveys,
          i.e. intensity mapping
     \end{itemize}
  }


\end{document}
